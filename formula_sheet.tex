\documentclass[10pt,letterpaper]{report}
\usepackage[latin1]{inputenc}
\usepackage{amsmath}
\usepackage{esint}
\usepackage[left=2cm, right=2cm, top=2cm, bottom=2cm]{geometry}
\usepackage{fancyhdr}
\pagestyle{fancy}
\pagenumbering{gobble}
\usepackage{array}

\begin{document}
  \fancyhead{} % clear all header fields
  \fancyhead[C]{\LARGE{\textbf{Electromagnetic Fields Formulas}}}
  \large

  \begin{equation*}
    \begin{split} % Left Column
      %
      % General Vector Equations
      %
      &\text{\underline{\textbf{General}}} \\
      &\textbf{A}\cdot \textbf{B} = ABcos(\theta_{AB}) \\ % dot product
      &\textbf{A} \times \textbf{B} = ABsin(\theta_{AB}) \\ % cross product
      &\textbf{A} \cdot (\textbf{B} \times \textbf{C}) \\ % scaler triple product
      &\begin{split}
       \quad &= \textbf{B} \cdot (\textbf{C} \times \textbf{A}) \\
             &= \textbf{C} \cdot (\textbf{A} \times \textbf{B} )  \\
             &=
         \begin{vmatrix}
           A_x & A_y & A_z \\
           B_x & B_y & B_z \\
           C_x & C_y & C_z
         \end{vmatrix} \\
      \end{split} \\
      &\textbf{A} \times \textbf{B} \times \textbf{C} \\ % vector triple product
      &\quad = \textbf{B}(\textbf{A} \cdot \textbf{C}) - \textbf{C}(\textbf{A} \cdot \textbf{B}) \\
      &A_B = \textbf{A} \cdot \textbf{a}_B = Acos\theta_{AB} \\ % scaler component of a on b
      &\textbf{A}_B = A_B\textbf{a}_B = (\textbf{A} \cdot \textbf{a}_B)\textbf{a}_B \\ % vector component of a on b
      &\oint_S \boldsymbol{A} \cdot d\boldsymbol{S} = \int_v \nabla \cdot \boldsymbol{A} dv \\
      &\oint_L \boldsymbol{A} \cdot d\boldsymbol{\ell} = \int_S (\nabla \times \boldsymbol{A}) \cdot d\boldsymbol{S} \\
      %
      % Cartesian Coordinate Equations
      %
      &\text{\underline{\textbf{Cartesian}}} \\
      &d^2 = (x_2 - x_1)^2 + (y_2 - y_1)^2 \\ % distance between points
      & \quad + (z_2 - z_1)^2 \\
      &\nabla = \frac{\partial}{\partial x}\textbf{a}_x +  \frac{\partial}{\partial y}\textbf{a}_y + \frac{\partial}{\partial z}\textbf{a}_z \\ % del
      &d\textbf{l} = dx \, \textbf{a}_x + dy \, \textbf{a}_y + dz \, \textbf{a}_z \\ % line integral dl
      &d\textbf{S} = dy \, dz \, \textbf{a}_x \, dx \, dz \, \textbf{a}_y \\ % surface integral dS
      & \quad dx \, dy \, \textbf{a}_z \\
      &dv = dx \, dy \, dz \\ % volume integral dv
      %
      % Cylindrical Coordinate Equations
      %
      &\text{\underline{\textbf{Cylindrical}}} \\
      &d^2 = \rho_1^2 + \rho_2^2 \\ % distance between points
      & \quad -2\rho_1\rho_2 cos( \phi_2 - \phi_1 ) \\
      &\quad + (z_2 - z_1)^2 \\
      &\lvert\textbf{A} \rvert = \sqrt{A_{\rho}^2 + A_{\phi}^2 + A_z^2} \\ % Magnitude of A
      %&\rho = \sqrt{x^2 + y^2} \\
      &\phi = tan^{-1} (y / x) \\
      &cos( \phi ) = x / \sqrt{x^2 + y^2} \\
    \end{split}
    \quad
    \begin{split} % Center Column
      &sin( \phi ) = y / \sqrt{x^2 + y^2} \\
      &d\textbf{l} = d\rho \, \textbf{a}_{\rho} \, \rho \, d\phi \textbf{a}_{\phi} \, dz \textbf{a}_z \\ % line integral dl
      &d\textbf{S} = \rho \, d\phi \, dz \, \textbf{a}_{\rho} \\ % Surface Integral dS
      & \quad \quad d\rho \, dz \textbf{a}_{\phi} \\
      & \quad \quad \rho \, d\rho \, d\phi \textbf{a}_{z} \\ %
      &dv = \rho \, d\rho \, d\phi \, dz \\ % Volume integral dV
      &x = \rho cos(\phi) \\
      &y = \rho sin(\phi) \\
      &z = z \\
      &\nabla = \textbf{a}_{\rho} \frac{\partial}{\partial \rho} + \textbf{a}_{\phi} \frac{1}{\rho} \frac{\partial}{\partial \phi} + \frac{\partial}{\partial z}\textbf{a}_z \\
      &\nabla \times \boldsymbol{A} = \frac{1}{\rho}
      \begin{vmatrix}
         \boldsymbol{a}_{\rho} & \rho \boldsymbol{a}_{\phi} & \boldsymbol{a}_z \\
         \frac{\partial}{\partial \rho} & \frac{\partial}{\partial \phi} & \frac{\partial}{\partial z} \\
         A_{\rho} & \rho A_{\phi} & A_z \\
      \end{vmatrix} \\
      &\nabla \cdot \boldsymbol{A} = \frac{1}{\rho} \frac{\partial}{\partial \rho}(\rho A_{\rho}) + \frac{1}{\rho} \frac{\partial A_{\phi}}{\partial \phi} + \frac{\partial A_z}{\partial z} \\
      &\begin{split}
        \nabla^2 V &= \frac{1}{\rho} \frac{\partial}{\partial \rho} \left(\rho \frac{\partial V}{\partial \rho}\right) \\
                   &+ \frac{1}{\rho^2} \frac{\partial^2 V}{\partial \phi^2} + \frac{\partial}{\partial z} \\
      \end{split} \\
      %
      % Spherical Coordinate Equations
      %
      &\text{\underline{\textbf{Spherical}}} \\
      &d^2 = r_2^2 + r_1^2 -2r_1r_2 cos \theta_2 cos \theta_1 \\ % distance between points
      &\quad -2r_1r_2 sin \theta_2 sin \theta_1 cos( \phi_2 - \phi_1 ) \\
      &x = r sin( \theta ) cos( \phi ) \\
      &y = r sin( \theta ) sin( \phi ) \\
      &z = r cos( \theta ) \\
      &\lvert \textbf{A} \rvert = \sqrt{A_{r}^2 + A_{\theta}^2 + A_{\phi}^2} \\ % Magnitude of A
      &\nabla = \frac{\partial}{\partial r} \textbf{a}_r + \frac{1}{r} \frac{\partial}{\partial \theta} \textbf{a}_{\theta} \frac{1}{r sin(\theta)} \frac{\partial}{\partial \phi} \\
     &r = \sqrt{x^2 + y^2 +z^2} \\
     &\theta = tan^{-1} (\sqrt{x^2 + y^2} /z ) \\
     &cos( \theta ) = z / \sqrt{x^2 + y^2 +z^2} \\
     &\phi = tan^{-1} (y / x) \\
    \end{split}
    \quad
    \begin{split} % Right Column
     &sin( \theta ) = \frac{\sqrt{x^2 + y^2}}{\sqrt{x^2 + y^2 +z^2}}\\
     &cos( \phi ) = x / \sqrt{x^2 + y^2} \\
     &sin( \phi ) = y / \sqrt{x^2 + y^2} \\
     &d\textbf{l} = dr\textbf{a}_r + r\, d\theta \textbf{a}_{\phi} \\
     &\quad + r sin(\theta) \, d\phi \textbf{a}_{\phi} \\
     &dv = r^2 \, sin( \theta )\, dr \, d\theta \, d\phi \\ % Volume integral dv
     &\begin{split}
        d\textbf{S} = &r^2 \, sin(\theta) \, d\theta \, d\phi \, \textbf{a}_r \\
                      &r\, sin(\theta) \, dr \, d\phi \, \textbf{a}_{\theta} \\
                      &r\, dr \, d\theta \textbf{a}_{\phi}
     \end{split} \\
     &\nabla = \frac{\partial}{\partial r}\boldsymbol{a}_r + \frac{1}{r} \frac{\partial}{\partial \theta}\boldsymbol{a}_{\theta} + \frac{1}{r sin(\theta)} \frac{\partial}{\partial \phi} \boldsymbol{a}_{\phi} \\
     &\nabla \times \boldsymbol{A} = \frac{1}{r^2 sin(\theta)}
     \begin{vmatrix}
        \boldsymbol{a}_r & r \boldsymbol{a}_{\theta} & \boldsymbol{a}_{\phi} \\
        \frac{\partial}{\partial r} & \frac{\partial}{\partial \theta} & \frac{\partial}{\partial \phi} \\
        A_r & rA_{\theta} & r sin(\theta) A_{\phi} \\
     \end{vmatrix} \\
     &\begin{split}
        \nabla \cdot \boldsymbol{A} &= \frac{1}{r^2} \frac{\partial}{\partial r}(r^2 A_r) \\
                                    &+ \frac{1}{r sin(\theta)} \frac{\partial}{\partial \theta}(A_{\theta} sin \theta)\\
                                    &+ \frac{1}{r sin(\theta)} \frac{\partial A_{\phi}}{\partial \phi} \\
     \end{split} \\
     &\begin{split}
        \nabla^2 V &= \frac{1}{r^2} \frac{\partial}{\partial r}\left(r^2 \frac{\partial V}{\partial r}\right) \\
                   &+ \frac{1}{r^2 sin(\theta)} \frac{\partial}{\partial \theta} \left(sin(\theta) \frac{\partial V}{\partial \theta}\right) \\
                   &+ \frac{1}{r^2 sin(\theta)} \frac{\partial^2 V}{\partial \phi^2}
     \end{split} \\
    %
    % Spherical Cylindrical Conversion Equations
    %
    &\text{\underline{\textbf{Spherical Cylindrical}}} \\
    &\rho = r sin( \theta ) \\
    &\phi = \phi \\
    &z = r cos( \theta ) \\
    &r = \sqrt{\rho^2 + z^2} \\
    &\theta = tan^{-1} \left(\frac{\rho}{z}\right)
    \end{split}
  \end{equation*}

  %
  % Coordinate System Conversion Matrices
  %
  \begin{equation*}
        %
        % Cartesian to Cylindrical
        %
        \begin{bmatrix}
          A_{\rho} \\
          A_{\phi} \\
          A_z
          \end{bmatrix}
          =
          \begin{bmatrix}
          cos \phi  & sin \phi & 0 \\
          -sin \phi & cos \phi & 0 \\
          0         & 0        & 1
          \end{bmatrix}
          \begin{bmatrix}
            A_x \\
            A_y \\
            A_z
          \end{bmatrix} \\
      % Right Column
      \quad \vline \quad
        %
        % Cylindrical to Cartesian
        %
        \begin{bmatrix}
           A_x \\
           A_y \\
           A_z
         \end{bmatrix}
         =
         \begin{bmatrix}
           cos \phi & -sin \phi & 0 \\
           sin \phi &  cos \phi & 0 \\
           0        &  0        & 1
         \end{bmatrix}
         \begin{bmatrix}
           A_{\rho} \\
           A_{\phi} \\
           A_z
         \end{bmatrix} \\
  \end{equation*}
  %
  % Coordinate System Conversion Matrices Comntinued
  %
  \begin{equation*}
      % Left Column
      \begin{split}
        %
        % Cartesian to Spherical
        %
        &\begin{bmatrix}
            A_r \\
            A_{\theta} \\
            A_{\phi}
          \end{bmatrix}
          =
          \begin{bmatrix}
            sin \theta cos \phi  & sin \theta sin \phi & cos \theta \\
            cos \theta cos \phi  & cos \theta sin \phi & -sin \theta \\
            -sin \phi            & cos \phi            & 0
          \end{bmatrix}
          \begin{bmatrix}
            A_x \\
            A_y \\
            A_z
          \end{bmatrix} \\
        %
        % Spherical to Cylindrical
        %
        &\begin{bmatrix}
          A_{\rho} \\
          A_{\phi} \\
          A_z
          \end{bmatrix}
          =
          \begin{bmatrix}
          sin \theta  & cos \theta  & 0 \\
          0           & 0           & 1 \\
          cos \theta  & -sin \theta & 0
          \end{bmatrix}
          \begin{bmatrix}
            A_r \\
            A_{\theta} \\
            A_{\phi}
          \end{bmatrix}
        \end{split}
        \quad \vline \quad
        %
        % Cylindrical to Spherical
        %
      \begin{split}
        %
        % Spherical to Cartesian
        %
        &\begin{bmatrix}
            A_x \\
            A_y \\
            A_z
          \end{bmatrix}
          =
          \begin{bmatrix}
            sin \theta cos \phi  & cos \theta cos \phi & -sin \phi \\
            sin \theta sin \phi  & cos \theta sin \phi & cos \phi \\
            cos \theta           & -sin \phi           & 0
          \end{bmatrix}
          \begin{bmatrix}
            A_r \\
            A_{\theta} \\
            A_{\phi}
          \end{bmatrix} \\
        &\begin{bmatrix}
            A_r \\
            A_{\theta} \\
            A_{\phi}
          \end{bmatrix}
          =
          \begin{bmatrix}
          sin \theta & 0 & cos \theta \\
          cos \theta & 0 & -sin \theta \\
          0          & 1 & 0
          \end{bmatrix}
          \begin{bmatrix}
            A_{\rho} \\
            A_{\phi} \\
            A_z
          \end{bmatrix}
    \end{split}
  \end{equation*}

  \begin{equation*}
    \begin{split} % Left Column
      %
      % Electro Static Fields
      %
      &\text{\underline{\textbf{Electrostatic Fields}}} \\
      &dQ = \rho_{l|S|v} \, dl|dS|dv \\
      &Q = \underset{L/S/v}{\int} \rho_{L|S|v} \, d\boldsymbol{l}|d\boldsymbol{S}|dv \\
      &\textbf{F}_{12} = \frac{Q_1Q_2}{4\pi\epsilon_0 R^2} \textbf{a}_{R_{12}} \\
      &\begin{split}
      \textbf{E} &= \frac{1}{4\pi\epsilon_0} \sum_{k=1}^N \frac{Q_k(\textbf{r} - \textbf{r}_k )}{|\textbf{r} - \textbf{r}_k |^3} \\
                 &= \underset{L/S/v}{\int} \frac{\rho_{L|S|v} \, d\boldsymbol{l}|d\boldsymbol{S}|dv}{4\pi\epsilon_0 R^2} \textbf{a}_r \\
                 &= \frac{\rho_S}{2 \epsilon_0} \textbf{a}_n \\ % Electric field intensity of infinite sheet charge
                 &= -\nabla V_{AB} \\
      \end{split} \\
      &\begin{split}
        \textbf{D} &= \epsilon_0 \textbf{E} \\ % electric flux density
                   &= \frac{\rho_S}{2} \textbf{a}_n \\
                   &= \int_v \frac{\rho_v \, dv}{4 \pi R^2} \textbf{a}_R \\
                   &= \varepsilon_0 \boldsymbol{E} + \boldsymbol{P} \\
      \end{split} \\
      &\begin{split}
        \Psi &= \int_S \textbf{D} \cdot d \textbf{S} \\ % flux
             &= Q_{\text{enc}} \\ % Gaus's Law
      \end{split} \\
      &\rho_v = \nabla  \cdot \textbf{D} \\ % Maxwell's
      &\begin{split}
        V(\textbf{r})
          &= \frac{1}{4\pi \epsilon_0} \sum_{k=1}^n \frac{Q_k}{|\textbf{r} - \textbf{r}_k|} \\ %Voltage point charges
          &= \frac{1}{4\pi \epsilon_0} \underset{L|S|v}{\int} \frac{\rho_{L|S|v} \, \textbf{r}' \, dl'|dS'|dv'}{|\textbf{r} - \textbf{r}'|} \\ % Voltage line/ surface/ volume charge
      \end{split} \\
      &V_{AB} = -\int_A^B \textbf{E} \cdot d\textbf{l} = \frac{W}{Q} \\
    \end{split}
    %
    % Column 2
    %
    \quad
    \begin{split}
      &\begin{split} % Work / Energy
        W_E &= \frac{1}{2} \sum_{k=1}^n Q_k V_k \\
            &= \frac{1}{2} \underset{L|S|v}{\int} \rho_{L|S|v} V \, d\boldsymbol{l}|d\boldsymbol{S}|dv \\ %\underset{L/S/v}{\rho}
            &= -\frac{1}{2} \int_v \textbf{D} \cdot \nabla V \, dv \\
            &= \frac{1}{2} \int_v (\textbf{D} \cdot \textbf{E}) \, dv \\
            &= \frac{1}{2} \int_v \epsilon_0 E^2 \, dv \\
            &= -Q \int_A^B \textbf{E} \cdot d\textbf{l} \\
      \end{split} \\
      &\oint_L \textbf{E} \cdot d\textbf{l} = \int_S (\nabla \times \textbf{E}) \cdot d\textbf{S} = 0 \\ % Maxwell
      &\nabla \times \textbf{E} = 0 \\
      &E = \frac{V}{\ell} \\
      &\begin{split}
        R &= \frac{V}{I} \\
          &= \frac{\ell}{\sigma S} \\
          &= \frac{\int_L \textbf{E} \cdot d\boldsymbol{\ell}}{\int_S \textbf{E} \cdot d\textbf{S}} \\
      \end{split} \\
      &I = \int_S \textbf{J} \cdot d\textbf{S} \\
      &\boldsymbol{P} = \lim_{\Delta v \to 0} \frac{\sum\limits_{K = 1}^{N} Q_K \boldsymbol{d}_K}{\Delta v} \\
      &\boldsymbol{p} = Q \boldsymbol{d} \\
      &\boldsymbol{P} = \chi_e \varepsilon_0 \boldsymbol{E} \\
    \end{split}
    %
    % Column 3
    %
    \quad
    \begin{split}
      &\varepsilon_r = 1 + \chi_e = \frac{\varepsilon}{\varepsilon_0} \\
      &\rho_{ps} = \boldsymbol{P} \cdot \boldsymbol{a}_n \\
      &\rho_{pv} = - \nabla \cdot \boldsymbol{P} \\
      &D_{1n} - D_{2n} = \rho_S \\
      &D_{1n} = D_{2n} \\
      &E_{1t} = E_{2t} = \frac{D_{1t}}{\varepsilon_1} = \frac{D_{2t}}{\varepsilon_2} \\
    \end{split}
  \end{equation*}
\end{document}